\documentclass[modern]{aastex62}

% \submitjournal{AJ}

\newcommand{\harvard}{Center for Astrophysics {\normalfont |} Harvard \& Smithsonian, 60
Garden Street, Cambridge, MA 02138, USA}
\newcommand{\nyuphys}{Center for Cosmology and Particle Physics, Department of
Physics, New York University, 726 Broadway, New York, NY 10003, USA}
\newcommand{\nyucds}{Center for Data Science, New York University, 60 5th
Ave., New York, NY 10011, USA}
\newcommand{\cca}{Center for Computational Astrophysics, Flatiron Institute,
162 5th Ave., New York, NY 10010, USA}
\newcommand{\mpia}{Max-Planck-Institut f\"ur Astronomie, K\"onigstuhl 17,
69117 Heidelberg, Germany}

\shorttitle{Distance Measures in Cosmology II.}
\shortauthors{Beane and Hogg}

\begin{document}

\title{Distance Measures in Cosmology II.}

\email{abeane@sas.upenn.edu, david.hogg@nyu.edu}

\author[0000-0002-8658-1453]{Angus Beane}
\affiliation{\harvard}

\author[0000-0003-2866-9403]{David W. Hogg}
\affiliation{\nyuphys}
\affiliation{\nyucds}
\affiliation{\cca}
\affiliation{\mpia}

\begin{abstract}

In cosmology (or to be more specific, {\em cosmography,\/} the measurement of
the Universe) there are many ways to specify the distance between two points,
because in the expanding Universe, the distances between comoving objects are
constantly changing, and Earth-bound observers look back in time as they look
out in distance. The unifying aspect is that all distance measures somehow
measure the separation between events on radial null trajectories, ie,
trajectories of photons which terminate at the observer. In this note,
formulae for many different cosmological distance measures are provided.  We
treat the concept of ``distance measure'' very liberally, so, for instance,
the lookback time and comoving volume are both considered distance measures.
The bibliography of source material can be consulted for many of the
derivations; this is merely a ``cheat sheet.'' Comments and corrections are
highly appreciated, as are acknowledgments or citation in research that makes
use of this summary or the associated code.

\end{abstract}

\keywords{editorials, notices --- 
miscellaneous --- catalogs --- surveys}

\section{Cosmographic Parameters} \label{sec:param}
The {\em Hubble constant\/} $H_0$ is the constant of proportionality between
recession speed $v$ and distance $d$ in the expanding Universe;
\begin{equation}
v=H_0\, d
\end{equation}
The subscripted ``0'' refers to the present epoch because in general $H$
changes with time.  The dimensions of $H_0$ are inverse time, but it is
usually written
\begin{equation}
H_0=100\,h~{\rm km\,s^{-1}\,Mpc^{-1}}
\end{equation}
where $h$ is a dimensionless number parameterizing our ignorance. (Word on the
street is that $0.66<h<0.76$.)  The inverse of the Hubble constant is the {\em
Hubble time\/} $t_{\rm H}$
\begin{equation}
\label{eq:th}
t_{\rm H}\equiv\frac{1}{H_0}
= 9.78\times10^9\,h^{-1}~{\rm yr}= 3.09\times10^{17}\,h^{-1}~{\rm s}
\end{equation}
and the speed of light $c$ times the Hubble time is the {\em Hubble
distance\/} $D_{\rm H}$
\begin{equation}
\label{eq:dh}
D_{\rm H}\equiv\frac{c}{H_0}
= 3000\,h^{-1}~{\rm Mpc}= 9.26\times10^{25}\,h^{-1}~{\rm m}
\end{equation}
These quantities set the scale of the Universe, and often cosmologists
work in geometric units with $c=t_{\rm H}=D_{\rm H}=1$.

The mass density $\rho$ of the Universe and the dark energy density $\rho_{\rm
de}$ are dynamical properties of the Universe, affecting the time evolution of
the metric, but in these notes we will treat them as purely kinematic
parameters. They can be made into dimensionless density parameters
$\Omega_{{\rm M}, 0}$ and $\Omega_{{\rm de}, 0}$ by
\begin{equation}
\Omega_{{\rm M}, 0}\equiv\frac{8\pi\,G\,\rho_0}{3\,H_0^2}
\end{equation}
\begin{equation}
\Omega_{{\rm de}, 0}\equiv\frac{8\pi\,G\,\rho_{\rm de}}{3\,H_0^2}
\end{equation}
(Peebles, 1993, pp~310--313), where the subscripted ``0''s indicate that the
quantities (which in general evolve with  time) are to be evaluated at the
present epoch. The time evolution of dark energy, governed by its equation of
state $w$, is given by
\begin{equation}
\Omega_{\rm de}(z) = \Omega_{{\rm de}, 0} (1+z)^{3(1+w)}\text{,}
\end{equation}
where $w$ can in general evolve with redshift. The standard cosmological
model, $\Lambda$CDM, is given by a constant $w=-1$.

Describing the evolution of dark energy in terms of its equation of state is
strictly a phenomological choice, and simply parametrizes the physical
mechanism behind dark energy. By only using the $w$-parametrization, we
implicitly ignore the great diversity of dark energy models (and even some
that are described by the $w$-parametrization). A useful though not exhaustive
list of different dark energy models is given in Table~1 of 1502.01590.

For many (even precision) late-time cosmology purposes (i.e., $z<20$), we can
ignore the energy contribution of radiation
\begin{equation}
\Omega_{{\rm r}, 0}\equiv\frac{8\pi\,G\,\rho_{{\rm r}, 0}}{3\,H_0^2}\text{,}
\end{equation}
since $\Omega_{{\rm r}, 0} \sim 9.2\times10^{-5}$.

A fourth density parameter $\Omega_k$ measures the ``curvature of space'' and
can be defined by the relation
\begin{equation}
\Omega_{\rm M}+\Omega_{\rm de}+ \Omega_{rm r} + \Omega_k= 1
\end{equation}
These parameters completely determine the geometry of the Universe if it is
homogeneous, isotropic, and matter-dominated.  By the way, the critical
density $\Omega=1$ corresponds to $7.5\times
10^{21}\,h^{-1}\,M_{\odot}\,D_{\rm H}^{-3}$, where $M_{\odot}$ is the mass of
the Sun.

It is a remarkable fact that in the twenty years since these notes were
originally published that the basic cosmological model is still consistent
with all current data. Today, the values of the fiducial cosmological model
are more or less set, with only one increasingly convincing hint of new
physics (the Hubble constant discrepancy). While measuring the cosmological
parameters to greater precision is interesting, most of the effort of current
surveys is directed towards gathering evidence that $\Lambda$CDM is not true.
Therefore, in this update, we will study three classes of cosmological models.

First, we will study three different ``vanilla'' $\Lambda$CDM models
\begin{center}
\begin{tabular}{lcc}
name & $\Omega_{{\rm M}, 0}$ & $\Omega_{\Lambda, 0}$ \\ \hline
fiducial & 0.3 & 0.7 \\
low density & 0.2 & 0.8 \\
high density & 0.4 & 0.6 \\
\end{tabular}
\end{center}
although the low density and high density models are excluded by current data.
Second, we will study three different cosmological models that allow for
curvature. While every class of inflation models predicts a nearly flat
universe, many predict some residual amount of curvature. If this curvature
can be detected, many different classes of inflation models will be excluded.
\begin{center}
\begin{tabular}{lccc}
name & $\Omega_{{\rm M}, 0}$ & $\Omega_{\Lambda, 0}$ & $\Omega_{k, 0}$ \\ \hline
fiducial & 0.3 & 0.7 & 0 \\
positive curved & 0.2 & 0.7 & 0.1 \\
negative curved & 0.4 & 0.7 & -0.1 \\
\end{tabular}
\end{center}
Again, the positive curved and negative curved models are excluded by current data.

\end{document}
