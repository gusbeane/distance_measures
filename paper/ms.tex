\documentclass[modern]{aastex62}

% \submitjournal{AJ}

\newcommand{\harvard}{Center for Astrophysics {\normalfont |} Harvard \& Smithsonian, 60
Garden Street, Cambridge, MA 02138, USA}
\newcommand{\nyuphys}{Center for Cosmology and Particle Physics, Department of
Physics, New York University, 726 Broadway, New York, NY 10003, USA}
\newcommand{\nyucds}{Center for Data Science, New York University, 60 5th
Ave., New York, NY 10011, USA}
\newcommand{\cca}{Center for Computational Astrophysics, Flatiron Institute,
162 5th Ave., New York, NY 10010, USA}
\newcommand{\mpia}{Max-Planck-Institut f\"ur Astronomie, K\"onigstuhl 17,
69117 Heidelberg, Germany}

\shorttitle{Distance Measures in Cosmology II.}
\shortauthors{Beane and Hogg}

\begin{document}

\title{Distance Measures in Cosmology II.}

\email{abeane@sas.upenn.edu, david.hogg@nyu.edu}

\author[0000-0002-8658-1453]{Angus Beane}
\affiliation{\harvard}

\author[0000-0003-2866-9403]{David W. Hogg}
\affiliation{\nyuphys}
\affiliation{\nyucds}
\affiliation{\cca}
\affiliation{\mpia}

\begin{abstract}

In cosmology (or to be more specific, {\em cosmography,\/} the measurement of
the Universe) there are many ways to specify the distance between two points,
because in the expanding Universe, the distances between comoving objects are
constantly changing, and Earth-bound observers look back in time as they look
out in distance. The unifying aspect is that all distance measures somehow
measure the separation between events on radial null trajectories, ie,
trajectories of photons which terminate at the observer. In this note,
formulae for many different cosmological distance measures are provided.  We
treat the concept of ``distance measure'' very liberally, so, for instance,
the lookback time and comoving volume are both considered distance measures.
The bibliography of source material can be consulted for many of the
derivations; this is merely a ``cheat sheet.'' Comments and corrections are
highly appreciated, as are acknowledgments or citation in research that makes
use of this summary or the associated code.

\end{abstract}

\keywords{editorials, notices --- 
miscellaneous --- catalogs --- surveys}


\end{document}
